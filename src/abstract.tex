\documentclass[../main.tex]{subfiles}
\graphicspath{{img/}}
\begin{comment}
    Briefly state the (1) research problem, (2) methodology, (3) key results, and (4) conclusion. Generally, abstracts are between 100 and 150 words--roughly 5-10 sentences.
\end{comment}
\begin{document}
I raggi cosmici sono una fonte naturale di particelle ad alta energia ($1 - $\SI{e19}{\GeV}) di origine galattica e extragalattica. 
I prodotti della loro interazione con l'atmosfera terrestre giungono fino alla superficie terrestre, dove vengono rilevati dagli esperimenti di fisica delle particelle come segnale di fondo. 
Si vuole quindi identificare e rimuovere questo segnale.
Gli apparati sperimentali usati in fisica delle particelle prevedono dei sistemi di selezione dei segnali in ingresso (detti \emph{trigger}) per rigettare segnali sotto una certa soglia di energia. 
Il progredire delle prestazioni dei calcolatori permette oggi di sostituire l'elettronica dei sistemi di \emph{trigger} con implementazioni software (\emph{triggerless}) in grado di selezionare i dati secondo criteri più complessi.
TriDAS (Triggerless Data Acquisition System) è un sistema di acquisizione \emph{triggerless} sviluppato per l'esperimento KM3NeT e utilizzato recentemente per gestire l'acquisizione di esperimenti di collisione di fascio ai Jefferson Lab (Newport News, VA). 
Il presente lavoro ha come scopo la definizione di un algoritmo di selezione di eventi generati da raggi cosmici e la sua implementazione come \emph{trigger} software all'interno di TriDAS.
Quindi si mostrano alcuni strumenti software sviluppati per costruire un ambiente di test del suddetto algoritmo e analizzare i dati prodotti da TriDAS. 
\end{document}
