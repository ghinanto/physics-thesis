\documentclass[../main.tex]{subfiles}
\graphicspath{{img/}}
\begin{comment}
    Discussion. Discuss the meaning of the results, stating clearly what their significance is. Compare the results with theoretical expectations and account for anything unexpected.

    Conclusions. Review the results in relation to the original problem statement. Assess the success of the study in light of the criteria of success you gave in the introduction.

    Recommendations. If applicable, recommend directions for future work.
\end{comment}

\begin{document}
L'algoritmo di identificazione di eventi generati da raggi cosmici implementato nel trigger L2 \emph{TrigFromBeam} è integrato con successo nel \emph{framework} di TriDAS. 

Allo stato attuale risultano delle incompatibilità nei sistemi di indici utilizzati dai diversi componenti software. 
Infatti, i PT file disponibili al momento per l'analisi sono quelli acquisiti ai Jefferson Lab utilizzando il rivelatore di cui nel capitolo 2 (matrice $3 \times 3$ di calorimetri). Il formato di dati utilizzato da TriDAS per l'acquisizione prevede una geometria di $23$ \say{torri}, ciascuna composta da $5$ \say{piani} di $16$ sensori ciascuno, per un totale di $1840$ sensori. Al momento dell'acquisizione quindi sono stati assegnati ai soli $9$ sensori dell'esperimento degli indici nell'intervallo $1 - 1840$. Inoltre gli indici non sono associati ai sensori secondo la disposizione dell'apparato sperimentale, per cui ad esempio i primi tre indici non corrispondono alla prima riga della matrice del rivelatore. 
È necessario quindi realizzare ulteriori analisi per rendere noto all'algoritmo l'associazione tra gli indici utilizzati da TriDAS e la posizione dei sensori all'interno della matrice del rivelatore. 

La continuazione di questo lavoro prevede la verifica dell'efficienza dell'algoritmo sfruttando il software \texttt{trig\_sim} descritto nel capitolo 3, utilizzando PT file precedentemente acquisiti e confrontandone il contenuto con quello dei nuovi PT file prodotti da \texttt{trig\_sim}.
A questo scopo possono essere utilizzati i programmi \texttt{pt2txt} e \texttt{plotxt}, anch'essi descritti nel capitolo 3.
\end{document}
