\documentclass[../main.tex]{subfiles}
\graphicspath{{\subfix{../img/}}}
\begin{document}
L'ampliarsi dell'orizzonte della ricerca scientifica a cui assistiamo quotidianamente non sarebbe possibile senza un'altrettanto quotidiana evoluzione degli strumenti di ricerca. I computer diventano sempre più capaci di accorciare i tempi di calcolo e sono ormai l'interfaccia necessaria fra l'essere umano e lo strumento di misura, in particolare negli ambiti della scienza come la fisica in cui la ricerca raggiunge i confini estremi della realtà separandosi completamente dall'esperienza comune. Tuttavia, i computer restano strumenti, il cui valore risiede nella mente che li ha programmati. Essi sono materia morta fino a che le dita creatrici del programmatore non gli infondono ordine e vita. In questo, per me, risiede tutto il fascino e la bellezza della ricerca tecnica ed è questo che mi ha mosso in questi mesi di lavoro. È una sfida con se stesso che l'uomo ingaggia, è una libera creazione che l'uomo intraprende.
\end{document}