\documentclass[../main.tex]{subfiles}
\graphicspath{{img/}}
\begin{comment}
    Introduction. State (1) the purpose of the investigation, (2) the problem being investigated, (3) the background (context and importance) of the problem (citing previous work by others), (4) your thesis and general approach, and (5) the criteria for your study's success.

    Theory. Develop the theoretical basis for your design or experimental work, including any governing equations. Detailed calculations go to an appendix.
\end{comment}
\begin{document}
In Fisica gli esperimenti ad alte energie sono quelli che ci permettono di esplorare le parti remote dell'universo, conoscere i comportamenti delle particelle e ricreare le condizioni in cui l'Universo si trovava nei primi istanti della sua vita. Per esempio, si realizzano esperimenti di collisione di fascio in cui delle particelle vengono accelerate e fatte scontrare su un bersaglio, per poi rilevare e analizzare i prodotti della collisione. Oppure si cerca di rilevare i prodotti degli eventi ad alte energie che già avvengono nel nostro universo, come le esplosioni di stelle o le collisioni di buchi neri. Buoni candidati per questa categoria sono i neutrini, particelle debolmente interagenti che attraversano quindi l'universo conservando l'informazione dell'evento che li ha generati. Proprio per questo però la loro rilevazione è estremamente ardua. L'esperimento europeo KM3Net si propone di farlo sfruttando la radiazione Cherenkov emessa dalle particelle cariche relativistiche generate dalla fortuita interazione tra i neutrini cosmici e l'acqua o i fondali marini. L'apparato del rilevatore quindi prevede torri di 18 unità ottiche ciascuna ancorate al fondale del Mar Adriatico al largo delle coste di Sicilia, Francia e Grecia, distribuite su un volume di un $km^3$ ciascuna. Ogni unità ottica comprende 31 fotomoltiplicatori (PMT) (Fig. \ref{fig:towers}). 

\begin{figure}[!b]
    \centering
    \includegraphics[width=0.5\textwidth]{KM3NetTowers.png}
    \caption{\small Centinaia di torri di 18 unità ottiche ciascuna ancorate al fondale e sostenute da un galleggiante vanno a formare l'installazione di KM3Net. Quando un neutrino interagisce con la materia terrestre in prossimità del volume del rilevatore, viene generato un muone che produce un cono di luce per effetto Cherenkov, il quale viene rilevato dai PMT.} \label{fig:towers}
\end{figure}       

    
    
    
    L'intero apparato genera quindi un flusso di dati di circa $30\ Gbps$, soprattutto a causa del contributo di rumore di fondo quali decadimenti di ${}^{40}K$, che per essere memorizzati e successivamente analizzati devono necessariamente essere filtrati impostando una soglia di energia. Se solitamente questa operazione avviene via hardware direttamente nel luogo dei sensori, in questo caso la locazione sottomarina impone un approccio \emph{off-shore}, dove il flusso di dati è trasmesso a terra e quindi processato via software. A questo scopo è stato sviluppato TriDAS (Trigger and Data Acquisition System), un sistema \emph{triggerless} (appunto che non prevede un trigger hardware) di acquisizione e filtraggio di dati. Con l'aumento delle capacità dei sistemi di calcolo, diventa sempre più comune implementare sistemi \emph{triggerless} via software. Per esempio, ai \emph{Jefferson Lab}, centro di ricerca statunitense di fisica nucleare (Newport News, Virginia) viene utilizzato TriDAS nel 2020 per processare i dati di un esperimento di collisione di fascio di elettroni. TriDas prevede anche la possibilità di eseguire altri livelli di trigger definiti dall'utente, caratteristica che lo rende molto versatile. Questo lavoro ha come scopo l'implementazione di un livello di trigger che escluda dall'acquisizione quei dati provenienti da fonti diverse dal fascio, i.e. raggi cosmici.

    Nel primo capitolo si mostra una breve descrizione degli esperimenti di collisione di fascio.
    Il secondo capitolo mostra la struttura e il funzionamento del software TriDAS.
    Quindi nel terzo capitolo si descrive come il trigger è stato implementato, i metodi usati per testarlo e l'apparato sperimentale su cui è stato testato, e i risultati ottenuti.
    Infine si commentano i risultati.

\end{document}