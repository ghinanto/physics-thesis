\usepackage[T1]{fontenc}
\usepackage[italian]{babel}
\usepackage{comment}
\usepackage{dirtytalk} 
\usepackage{import}
\usepackage{graphicx}
\graphicspath{{img/}}
\usepackage[margin=10pt, labelfont=bf, font=small, labelsep=quad]{caption}
%\usepackage{flafter}  
%\usepackage{placeins}  
\usepackage{blindtext}
\usepackage{xcolor}
\usepackage{geometry}
\usepackage{lipsum}
\usepackage[colorlinks=true,
linkcolor=black,
urlcolor=teal,
citecolor=black]{hyperref}
\usepackage{siunitx}
\usepackage{biblatex}
\addbibresource{citation.bib}
\usepackage{url}
\newcommand{\dd}[1]{\mathrm{d}#1}
\usepackage[libertine]{newtxmath}
%\usepackage{newtxtext,newtxmath}
\usepackage{inconsolata}
\usepackage{listings}

\usepackage{subfiles} % Best loaded last in the preamble

\geometry{
    paper=a4paper, % Change to letterpaper for US letter
    inner=2.5cm, % Inner margin
    outer=3cm, % Outer margin
    bindingoffset=0.5cm, % Binding offset
    top=2.5cm, % Top margin
    bottom=3cm, % Bottom margin
    %showframe, % Uncomment to show how the type block is set on the page
}

\hypersetup{
    colorlinks=true,
    linkcolor=black,
    filecolor=magenta,      
    urlcolor=cyan,
    %pdftitle={Overleaf Example},
    %pdfpagemode=FullScreen,
}

%Custom 
\definecolor{plainText}{RGB}{74,131,31}
\definecolor{comments}{RGB}{74,131,31}
\definecolor{strings}{RGB}{180,54,34}
\definecolor{numbers}{RGB}{44,45,211}
\definecolor{keywords}{RGB}{160,49,158}
\definecolor{preprocessorStatements}{RGB}{109,75,48}
\definecolor{classNames}{RGB}{101,63,165}
\definecolor{background}{RGB}{245,248,250}

\lstdefinestyle{c++}{
  frame=single,
  %frameround=tttt,
  backgroundcolor=\color{background},
  language=C++,
  % You need to declare < and > as "letters" in order to highlight
  % words that contain those characters:
  alsoletter=<>,
  % You can use multiple classes of words to emphasize,
  % each with its own style, as follows:
  %emph      = {[1]cout, for, int, std},
  %emphstyle = {\bfseries},
  %emph      = {[2]double, std},
  %emphstyle = {[2]\color{blue}\bfseries},
  % The style of compiler directives can be customise via...
  directivestyle=\color{magenta}\itshape, % (for instance)
  aboveskip=4mm,
  belowskip=4mm,
  showstringspaces=false,
  columns=flexible,
  basicstyle={\small\ttfamily},
  numbers=none,
  numberstyle=\tiny\ttfamily\color{black},
  keywordstyle=\bfseries\color{keywords},
  commentstyle=\color{comments},
  stringstyle=\color{strings},
  breaklines=true,
  breakatwhitespace=true, % (a comma was missing there)
  tabsize=4,
}

\lstdefinelanguage{json}{
    string=[s]{"}{"},
    stringstyle=\color{magenta}\itshape,
    comment=[l]{:},
    commentstyle=\color{black},
}

\lstdefinelanguage{docker-compose-2}{
    %string=[s]{\ }{:},
    %stringstyle=\color{magenta}\itshape,
    basicstyle=\color{magenta}\ttfamily,
    comment=[l]{:},
    morecomment=[l]{\-\ },
    commentstyle=\color{black},
}

\renewcommand{\lstlistingname}{Listato}
\lstset{style=c++}
\linespread{1}
