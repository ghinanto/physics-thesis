\documentclass[aspectratio=169]{beamer}
\usetheme[progressbar=none]{metropolis}           % Use metropolis theme
%\input{metropolis-settings.tex}
\usepackage[T1]{fontenc}
\usepackage[italian]{babel}
\usepackage[sfdefault]{FiraSans}
\usepackage{graphicx}
\graphicspath{{../img/}}

\title{Algoritmo di tagging di raggi cosmici\\in un sistema di acquisizione triggerless\\\mbox{per un esperimento di fisica di particelle su fascio}}
\date{Anno Accademico 2022/2023}
\author{Antonio Ghinassi}
\institute{Alma Mater Studiorum $\cdot$ Università di Bologna}

\begin{document}
  \maketitle
  \setbeamertemplate{frame footer}{\insertinstitute}%Alma Mater Studiorum $\cdot$ Università di Bologna}
  %\section{First Section}
  \begin{frame}{Raggi cosmici}
    
\begin{columns}[onlytextwidth,T]
  \begin{column}{.45\linewidth}
      Intensità muoni cosmici di \mbox{energia $\small>$\SI{1}{\GeV}} al livello del mare:
      $$
          I \sim 70\ m^{-2} s^{-1} sr^{-1}
      $$
      Contenuto di energia dei raggi cosmici:
      %\[
      $$
      1 - \SI{10e10}{\GeV}
      $$
      %\]
  \end{column}
  %\begin{column}{.45\linewidth}
  %\end{column}
\end{columns}
    % create TikZ picture environment
    \begin{tikzpicture}[remember picture, overlay]
    \node[left=1.2cm] at (current page.east) 
    {
        \includegraphics[width=0.5\textwidth]{cosmo-cascade.png}
    };
    \end{tikzpicture}

  \end{frame}
  \begin{frame}{TriDAS}
      \begin{itemize}
         \item bau
         \item ciao
      \end{itemize}
    Hello, world!
  \end{frame}
  \begin{frame}{TriDAS @ JLab}
    Hello, world!
  \end{frame}
  \begin{frame}{Algoritmo \texttt{I}}
    Hello, world!
  \end{frame}
  \begin{frame}{Algoritmo \texttt{II}}
    Hello, world!
  \end{frame}
  \begin{frame}{Algoritmo \texttt{III}}
    Hello, world!
  \end{frame}
  \begin{frame}{Strumenti di analisi e test}
    Hello, world!
  \end{frame}
  \begin{frame}{Conclusioni}
    Hello, world!
  \end{frame}
  \begin{frame}[standout]
      Grazie per l'attenzione.
  \end{frame}
\end{document}
