\documentclass[aspectratio=169]{beamer}
\usetheme[progressbar=none]{metropolis}           % Use metropolis theme
%\input{metropolis-settings.tex}
\usepackage[T1]{fontenc}
\usepackage{roboto}
%\setmathfont{Hack}
\usepackage{mathtools}
\usepackage[mathrm=sym]{unicode-math}
\setmathfont{Fira Math}
\usepackage[italian]{babel}
%\usepackage[sfdefault]{FiraSans}
%\usepackage{FiraCode}
\setmonofont{Fira Code}
%\usepackage{fourier}
%\usepackage{kpfonts}
%\usepackage{noto}
%\usepackage{tgpagella}
%\setsansfont[BoldFont={FiraSans-SemiBold.ttf}]{FiraSans-Light.ttf}
%\setsansfont{kpfonts}
%\usepackage{amsmath}
\usepackage{graphicx}
\graphicspath{{../img/}}
\usepackage{siunitx}
\usepackage{listings}
\usepackage{comment}
%\usepackage{amsmath, amssymb, dsfont}
\usepackage{cases}
%Custom 
\definecolor{plainText}{RGB}{74,131,31}
\definecolor{comments}{RGB}{74,131,31}
\definecolor{strings}{RGB}{180,54,34}
\definecolor{numbers}{RGB}{44,45,211}
\definecolor{keywords}{RGB}{160,49,158}
\definecolor{preprocessorStatements}{RGB}{109,75,48}
\definecolor{classNames}{RGB}{101,63,165}
\definecolor{background}{RGB}{245,248,250}

\lstdefinestyle{c++}{
  frame=single,
  %frameround=tttt,
  backgroundcolor=\color{background},
  language=C++,
  % You need to declare < and > as "letters" in order to highlight
  % words that contain those characters:
  alsoletter=<>,
  % You can use multiple classes of words to emphasize,
  % each with its own style, as follows:
  %emph      = {[1]cout, for, int, std},
  %emphstyle = {\bfseries},
  %emph      = {[2]double, std},
  %emphstyle = {[2]\color{blue}\bfseries},
  % The style of compiler directives can be customise via...
  directivestyle=\color{magenta}\itshape, % (for instance)
  aboveskip=4mm,
  belowskip=4mm,
  showstringspaces=false,
  columns=flexible,
  basicstyle={\small\ttfamily},
  numbers=none,
  numberstyle=\tiny\ttfamily\color{black},
  keywordstyle=\bfseries\color{keywords},
  commentstyle=\color{comments},
  stringstyle=\color{strings},
  breaklines=true,
  breakatwhitespace=true, % (a comma was missing there)
  tabsize=4,
}

\lstdefinelanguage{json}{
    string=[s]{"}{"},
    stringstyle=\color{magenta}\itshape,
    comment=[l]{:},
    commentstyle=\color{black},
}

\lstdefinelanguage{docker-compose-2}{
    %string=[s]{\ }{:},
    %stringstyle=\color{magenta}\itshape,
    basicstyle=\color{magenta}\ttfamily,
    comment=[l]{:},
    morecomment=[l]{\-\ },
    commentstyle=\color{black},
}

\renewcommand{\lstlistingname}{Listato}
\lstset{style=c++}

\title{Algoritmo di tagging di raggi cosmici\\in un sistema di acquisizione triggerless\\\mbox{per un esperimento di fisica di particelle su fascio}}
\date{Anno Accademico 2022/2023}
\author{Antonio Ghinassi}
\institute{Alma Mater Studiorum $\cdot$ Università di Bologna}

\begin{document}
  \maketitle
  \setbeamertemplate{frame footer}{\insertinstitute}%Alma Mater Studiorum $\cdot$ Università di Bologna}
  %\section{First Section}
  \begin{frame}{Raggi cosmici}
    
\begin{columns}[onlytextwidth,T]
  \begin{column}{.45\linewidth}
      Intensità muoni cosmici di \mbox{energia $\small>$\SI{1}{\GeV}} al livello del mare:
      $$
          I \sim 70\ m^{-2} s^{-1} sr^{-1}
      $$
      Contenuto di energia dei raggi cosmici:
      %\[
      $$
      1 - \SI{10e10}{\GeV}
      $$
      %\]
  \end{column}
  %\begin{column}{.45\linewidth}
  %\end{column}
\end{columns}
    % create TikZ picture environment
    \begin{tikzpicture}[remember picture, overlay]
    \node[left=1.2cm] at (current page.east) 
    {
        \includegraphics[width=0.5\textwidth]{cosmo-cascade.png}
    };
    \end{tikzpicture}

  \end{frame}
  \begin{frame}{TriDAS}
      \begin{itemize}
         \item bau
         \item ciao
      \end{itemize}
    Hello, world!
  \end{frame}
  \begin{frame}{TriDAS @ JLab}
    Hello, world!
  \end{frame}
  \begin{frame}{Algoritmo \texttt{I}}
    Hello, world!
  \end{frame}
  \begin{frame}{Algoritmo \texttt{II}}
    Hello, world!
  \end{frame}
  \begin{frame}{Algoritmo \texttt{III}}
    Hello, world!
  \end{frame}
  \begin{frame}{Strumenti di analisi e test}
    Hello, world!
  \end{frame}
  \begin{frame}{Conclusioni}
    Hello, world!
  \end{frame}
  \begin{frame}[standout]
      Grazie per l'attenzione.
  \end{frame}
\end{document}
